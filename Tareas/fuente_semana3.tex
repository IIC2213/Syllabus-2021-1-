\documentclass[12pt]{article}
\usepackage[utf8]{inputenc}
\usepackage[dvips]{graphics}
\usepackage[spanish]{babel}
\usepackage{amsmath}
\usepackage{amssymb}
\usepackage{fullpage}
\usepackage{epsfig}
\usepackage{multicol}
\usepackage{hyperref}

%\input{logo.tex}

\renewcommand{\L}{L}
\newcommand{\A}{\textbf{A}}
\newcommand{\N}{\mathbb{N}}
\newcommand{\join}{\bowtie}
\renewcommand{\epsilon}{\varepsilon}
\renewcommand{\phi}{\varphi}
\newcommand{\modelos}{\textit{modelos}}
\newcommand{\ModProx}{\textit{ModProx}}
%\thispagestyle{empty}
\pagestyle{empty}

%\parindent 0pt
%\parskip 0pt

\begin{document}

%\logo

\vspace*{5mm}
\begin{center}
{ IIC2213 --- Lógica para Ciencia de la Computación} \\
\vspace{3mm}
{\Large\bf Notas Semana 3} \\
\vspace{2mm}
\end{center}



%\begin{flushright} fecha fecha fecha \end{flushright}

\section{Lenguajes indecidibles, reducciones}

En esta clase nos vamos a adentrar más en poder mostrar qué lenguajes son indecidibles y cuáles no. Estas propiedades son fundamentales 
al tomarlas en conjunto con la hipótesis de Church-Turing: nos dicen qué cosas pueden o no pueden hacer los computadores. 

Lo primero que vamos a hacer es mostrar la existencia de dos lenguajes indecidibles. Sabemos por el argumento de conteo que no puede ser que todos los lenguajes sean 
decidibles (hay una cantidad enumerable de máquinas de turing, pero una cantidad no-numerable de lenguajes). Sin embargo, la demostración de abajo es \emph{constructiva} 
(construye los lenguajes), y por lo tanto más valiosa, por que esos lenguajes van a ser nuestro punto de partida. Otra gracia es que los lenguajes de abajo son recursivamente enumerables, 
y por tanto esto también muestra la existencia de lenguajes RE que no son decidibles. 

\subsection{Diagonalización}

Consideremos el lenguaje 
\begin{multline*}
D = \{w \in \{0,1\}^* \mid \\ 
\text{ existe una máquina } M \text{ tal que } w = C(M) \text{ y } M \text{ se detiene con entrada } w\}
\end{multline*}

En palabras, $D$ es el lenguaje de todas las codificaciones de máquinas tales que la máquina codificada se detiene con el input dado por su propia codificación. 

\medskip
\noindent
\textbf{Teorema} (diagonalización). El lenguaje $D$ es indecidible.

Este teorema se demuestra usando la técnica de diagonalización de Cantor.  Podemos también usar la misma técnica para demostrar que el siguiente lenguaje también es indecidible: 
\begin{multline*}
DA = \{w \in \{0,1\}^* \mid \text{ existe una máquina } M \text{ tal que } w = C(M) \text{ y } M \text{ acepta a } w\}
\end{multline*}


\subsection{Reducciones}


Queremos demostrar que $H$ es indecidible, pero hasta ahora solo sabemos que $D$ es indecidible. 
Sería un poco incómodo tener que encontrar un argumento parecido a diagonalización cada vez que queremos demostrar que un lenguaje es indecidible. 
En vez de eso, vamos a usar el concepto de 
\emph{Reducción de Turing}, que nos va a ayudar a capturar la idea de que cierto lenguaje $L_2$ es al menos tan difícil como un lenguaje $L_1$.
Recuerda que las funciones computables son todas las que pueden ser expresadas por máquinas de Turing. 

\medskip
\noindent
\textbf{Definición}. Dados lenguajes $L_1$ y $L_2$ sobre un alfabeto $\A$, decimos que $L_1$ es Turing-reducible a $L_2$, 
o igualmente que existe una reducción (de Turing) de $L_1$ a $L_2$, si existe una función computable $f$ tal que para todo $w \in \A^*$, 
$$w \in L_1 \text{ si y solo si } f(w) \in L_2.$$ 

Como prometimos, el concepto de reducción captura la noción de que $L_2$ es al menos tan difícil como $L_1$, o que, equivalentemente, 
$L_1$ no es más difícil que $L_2$. Esto lo podemos formalizar así: 

 \medskip
\noindent
\textbf{Teorema}[1.2]
\label{teo-reduccion}. Si $L_1$ es reducible a $L_2$, entonces:
\begin{itemize}
\item Si $L_2$ es decidible, entonces $L_1$ es decidible. 
\item Si $L_1$ es indecidible, entonces $L_2$ es indecidible
\end{itemize}

\medskip
\noindent
\textbf{Demostración}. Como $L_1$ es reducible a $L_2$ entonces existe una función computable $f$ y una 
máquina $M_r$ que, al recibir 
como input una palabra $w \in L_1$ se detiene con $f(w)$ en la cinta. 

Supongamos primero que $L_2$ es decidible, y sea $M_2$ una máquina de turing tal 
que $L_2 = L(M_2)$. Para mostrar que $L_1$ es decidible tenemos que construit una máquina $M_1$ tal que 
$L_1 = L(M_1)$. Definimos la máquina de la siguiente manera: 

\smallskip
\noindent
\textit{
Con input $w$, la máquina $M_1$ primero ejecuta las instrucciones de $M_r$. Cuando $M_r$ se detiene, ejecuta luego las 
instrucciones de $M_2$, y se detiene cuando se detiene $M_2$. En otras palabras, para construir $M_1$, 
juntamos $M_r$ y $M_2$ y reemplazamos el estado inicial de $M_2$ por el final de $M_r$: 
si $M_r = (Q_r,\A,q_0^r,q_f^r,\delta_r)$ y $M_2 = (Q_2,\A,q_0^2,F_2,\delta_2)$, entonces 
$M_1 = (Q_r \cup Q_2^*,\A,q_0^r,F_2,\delta_r \cup \delta_2^*)$, pero en donde $Q_2^*$ y $\delta_2^*$ 
corresponden a reemplazar $q_0^2$ por $q_f^r$. }

\smallskip
Para mostrar que $L_1 = L(M_1)$ mostramos ambos lados de la inclusión. Primero, para ver que $L_1 \subseteq L(M_1)$, 
sea $w \in L_1$. Con input $w$, $M_r$ computa $f(w)$, y por la propiedad de las reducciones sabemos que 
$f(w) \in L(M_2)$. Por construcción de $M_1$ deducimos que $w \in L(M_1)$ (por que $M_1$ es equivalente a correr $M_r$ y sobre eso $M_2$). Para ver que $L(M_1) \subseteq L_1$, 
sea $w \in L(M_1)$. Con input $w$, $M_r$ siempre computa $f(w)$. Luego si $w \in L(M_1)$ entonces 
$f(w)$ debe pertenecer al lenguaje de $M_2$. Por la propiedad de las reducciones, y como asumimos que $L_2 = L(M_2)$, 
obtenemos que $w$ pertenece a $L_1$. 

\medskip

Supongamos ahora que $L_1$ es indecidible, y para obtener una contradicción, que $L_2$ es decidible. 
Usando el argumento anterior, del hecho que $L_2$ es decidible podemos deducir que $L_1$ es decidible, lo que es una 
contradicción, y por ende $L_2$ debe ser indecidible (¿puedes reproducir este argumento con el mismo detalle de arriba?)


\subsection*{Problema de la parada es indecidible}

Usando el teorema anterior podemos demostrar que $H$ es indecidible. 
Sea $\A = \{0,1,\#\}$ y $f$ una función de $\A^*$ en $\A^*$ definida como 
$f(w) = w\#w$. 

\medskip
\noindent
\textbf{Ejercicio}. Demuestra que $w \in D$ si y solo si $f(w) \in H$. Usa el teorema \ref{teo-reduccion} para concluir que $H$ es indecidible. 


\section{Relaciones entre lenguajes recursivamente enumerables e indecidibles}

Claramente, todo lenguaje decidible también es recursivamente enumerable (RE), pero el teorema de diagonalización nos muestra que hay lenguajes RE que no son decidibles. 
Podemos explicar como hacer máquinas de turing para mostrar: 

\medskip
\noindent
\textbf{Propiedad 1}. Si $L_1$ y $L_2$ son lenguajes decidibles, entonces $L_1 \cap L_2$, $L_1 \cup L_2$ y el complemento $\bar{L_1}$ son todos decidibles. 

\bigskip
Esta propiedad se demuestra construyendo las máquinas adecuadas. Un poco más complicado, pero podemos mostrar también:

\medskip
\noindent
\textbf{Propiedad 2}. Si $L_1$ y $L_2$ son lenguajes RE, entonces $L_1 \cap L_2$ y $L_1 \cup L_2$ son RE. 

\bigskip

Y qué pasa con el complemento de un lenguaje RE? Su complemento no necesariamente es RE. De hecho: 

\medskip
\noindent
\textbf{Propiedad 3}. Si $L$ es RE, y el complemento $\bar L$ es RE, entonces $L$ es decidible. 

\medskip
\noindent
\textbf{Demostración}. Por definición tenemos una máquina $M_L$ que acepta a una palabra $w$ si y solo si $w$ pertenece a $L$, y una  
una máquina $M_{\bar L}$ que acepta a una palabra $w$ si y solo si $w$ pertenece a $\bar{L}$. Luego, podemos construir una máquina $M$ que funcione de la siguiente forma: 
con input $w$, $M$ echa a correr $M_L$ y $M_{\bar L}$ \emph{en paralelo}, una instrucción a la vez. Si $M_L$ acepta, aceptamos, si $M_{\bar L}$ acepta, $M$ rechaza. 
Si $w \in L$ entonces sabemos que la componente $M_L$ de $M$ va a parar, y si 
$w \notin L$ significa que $w \in \bar L$, y por tanto la componente $M_{\bar L}$ va a parar. Concluimos que $M$ acepta a $L$, y se detiene en todas sus entradas, por lo que $L$ es decidible. 

\medskip

En general, la clase de lenguajes cuyo complemento es RE se conoce como coRE, y por la propiedad anterior vemos que RE intersectado con coRE corresponde a la clase 
de lenguajes decidibles. 


\section{Más lenguajes indecidibles}

Demostremos que esto es indecidible
\begin{multline*}
DyD = \{w \in \{0,1\}^* \mid  
\text{ existen máquinas } M_1,M_2 \text{ tal que } w = C(M_1)0000C(M_2), \text{y} \\ M_1 \text{ se detiene con entrada } C(M_1), M_2 \text{ se detiene con entrada } C(M_2) \}
\end{multline*}

\medskip
\noindent
\textbf{Solución}. Reduciendo desde $D$, simplemente haciendo $f(w) = w0000w$. Esta función es computable, y es fácil ver que $w$ pertenece a $D$ si y solo si 
$f(w)$ pertenece a $DyD$. 


\medskip
\noindent
\textbf{Ejercicio}. Muestra la indecidiblidad de 
\begin{multline*}
DoD = \{w \in \{0,1\}^* \mid 
\text{ existen máquinas } M_1,M_2 \text{ tal que } w = C(M_1)0000C(M_2) \text{ y} \\ \text{o } M_1 \text{ se detiene con entrada } C(M_1) \text{ o } M_2 \text{ se detiene con entrada } C(M_2) \}
\end{multline*}

\subsection{Reducciones más complejas}

Muestra la indecibilidad de: 
$$
SIM = \{w \in \{0,1\}^* \mid \text{ existe una MT } M \text{ tal que } w = C(M)0000v \text{ y } M \text{ acepta a } v\}
$$

\medskip
\noindent
\textbf{Solución}. Vamos a reducir desde $H$. Sea $f$ la siguiente función. 
Si $w$ corresponde a la codificación $C(M)0000v$ de una máquina $M$ y un string $v$, $f(w)$ es la codificación $C(M')0000v$ de una máquina 
$M'$ que es igual a $M$ solo que todos sus estados son finales y el mismo string $v$. Si $w$ no corresponde a esa codificación, entonces $f(w) = C(\hat M)00001$, donde 
$\hat M$ es una maquina con lenguaje 
vacío. 

Demuestra estas dos cosas, con lo que puedes invocar al teorema de reducciones. 
\begin{enumerate}
\item Comprueba que $w$ pertenece a $D$ si y solo si $f(w)$ pertenece a $SIM$. 
\item Que $f$ es computable por una máquina de turing. 
\end{enumerate}


Veamos ahora que este lenguaje también es indecidible: 
\begin{multline*}
ALL = \{w \in \{0,1\}^* \mid \\ 
\text{ existe una MT } M \text{ tal que } w = C(M) \text{ y } M \text{ acepta todas las palabras}\}
\end{multline*}


\medskip
\noindent
\textbf{Solución}. Vamos a reducir desde $D$. 
Sea $f$ la siguiente función. Si $w$ corresponde a la codificación $C(M)$ de una máquina $M$, $f(w)$ es la codificación $C(M')$ de una máquina 
$M'$ que hace lo siguiente: Con cualquier input, $M'$ borra toda la cinta, la reemplaza por $w$, y luego simula a $M$ con input $w$, con la excepción de que todos los 
estados de $M$ son estados finales. 
Si $w$ no corresponde a esa configuración, entonces $f(w)$ es la codificación de una máquina con lenguaje 
vacío. 

Tenemos que comprobar dos cosas: 
\begin{enumerate}
\item Comprobar que $w$ pertenece a $D$ si y solo si $f(w)$ pertenece a $ALL$. 
\item Que $f$ es computable por una máquina de turing. 
\end{enumerate}

Vemos (1). Supongamos primero que $w$ pertenece a $D$, luego en particular $w = C(M)$ para alguna máquina $M$. 
Entonces, $f(w)$ es la codificación de una máquina que siempre borra el input, y echa a andar a $M$ con input $C(M)$. Si $w$ pertenece a $D$, entonces 
$M$ se detiene con input $C(M)$, y por tanto $f(w)$ es la codificación $C(M')$ de una máquina $M'$ que siempre va a aceptar cualquier input, pues siempre va a detenerse en un estado final. 
Si $f(w)$ pertenece a ALL, eso quiere decir que $f(w) = C(M')$ y $M'$ acepta a todas las palabras. Esto implica que el lenguaje de $M'$ no es vacío, y de acuerdo a 
la construcción de $f(w)$, tiene que ser que la palabra $w$ corresponda a la codificación de una máquina, es decir, $w = C(M)$, y en ese caso $M'$ es una máquina que llama a $M$ con input $C(M)$. Como $M'$ acepta a todas las palabras, debe ser que $M$ se detiene con input $C(M)$, y por tanto $w$ pertenece a $D$. 

\smallskip

Para (2), lo primero es chequear si $w$ corresponde a la codificación de una máquina, lo que es computable (es un parser). Luego, si $w$ es $C(M)$, $f(w)$ tiene que tomar $M$ y construir $M'$ agregándole a $M$ la funcionalidad de borrar todo el input para correr en vez con input $C(M)$, y luego hacer todos los estados de $M$ finales. Todo esto puede ser construido directamente desde la codificación; en este curso no vamos a pedir mayor detalle que esto. Si $w$ no es la codificación, entonces cualquier máquina con lenguaje vacío basta. 

\end{document}
